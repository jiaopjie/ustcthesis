%% USTC thesis template ------------------------------------------
\documentclass[doctor]{ustcthesis}
%% 模板有以下三个文档选项, 各选项中的第一个选项为默认选项
%% [doctor|master|bachelor]
%% [academic|professional]
%% [chinese|english]
%%
%% book 文档类原有的文档选项继续有效
%% 默认 twoside, openany

%% Package -------------------------------------------------------
\usepackage{enumitem}
\setlist{nolistsep, leftmargin=*}
\setlist[1]{labelindent=\parindent}
\setlist[enumerate,1]{label=\textup{(\arabic*)}}

\usepackage{amsmath,amssymb}
\usepackage{longtable,booktabs}

%% Theorems ------------------------------------------------------
\usepackage{amsthm,aliascnt}
\def\proofname{\upshape\sffamily\indent 证.}
\newtheoremstyle{ustctheorem}
  {\topsep}{\topsep}{\itshape}{\parindent}{\sffamily}{}{1em}{}
\newtheoremstyle{ustcdefinition}
  {\topsep}{\topsep}{}{\parindent}{\sffamily}{}{1em}{}
\theoremstyle{ustctheorem}
\newtheorem{theorem}{定理}[section]
\theoremstyle{ustcdefinition}
\newaliascnt{definition}{theorem}
\newtheorem{definition}[definition]{定义}
\aliascntresetthe{definition}
\newcommand\definitionautorefname{定义}
\newaliascnt{remark}{theorem}
\newtheorem{remark}[remark]{注}
\aliascntresetthe{remark}
\newcommand\remarkautorefname{注}

%% Macro ---------------------------------------------------------
\newcommand\Emph[1]{\textsf{#1}}

%% Customize -----------------------------------------------------
%\hypersetup{colorlinks}       % 超链接以彩色显示
\bibliographystyle{plainnat}  % 参考文献列表样式

%%%  扉页中显示密级需要仿宋字体, 显示专业学位类型需要隶书字体
% \providecommand\fangsong{\CJKfontspec{FangSong}}
% \providecommand\lishu{\CJKfontspec{LiSu}}
%%%  更改中文字体为思源字体
% \setCJKmainfont[ItalicFont={KaiTi}]{Source Han Serif SC}
% \setCJKsansfont[ItalicFont={KaiTi}]{Source Han Sans SC}
%%%  使用中易宋体和黑体, 并开启伪粗体
 \setCJKmainfont[AutoFakeBold=3,ItalicFont={KaiTi}]{SimSun}
 \setCJKsansfont[AutoFakeBold=3,ItalicFont={KaiTi}]{SimHei}

%% Title ---------------------------------------------------------
\title{中国科学技术大学\texorpdfstring{\\}{}学位论文模板示例文档}
\author{赵钱孙}
\major{某}
\supervisor{周吴郑~教授\quad 冯陈褚~教授}
\date{\ctexset{today=big}\today}      % 默认是当前日期, 可手动修改
\professionaltype{专业学位类型}
%\secrettext{机密\quad 小于等于20年}   % 内部|秘密|机密, 注释本行则不保密

\entitle{An example of thesis template for
  University of Science and Technology of China}
\enauthor{Qiansun Zhao}
\enmajor{Whatever}
\ensupervisor{Prof.~Wuzheng Zhou, Prof.~Chenchu Feng}
\endate{\ctexset{today=old}\today}    % default is today
\enprofessionaltype{Professional degree type}
%\ensecrettext{Confidential\quad Less than or equal to 20 years}  % Internal|Secret|Confidential

%% ---------------------------------------------------------------
\begin{document}
%% ---------------------------------------------------------------
%%
%% 本科生论文文档顺序:
%%   front matter: 致谢、目录、中文摘要、英文摘要
%%   main matter:  正文、参考文献
%%   appendix:     附录
%%
%% 研究生论文文档顺序:
%%   front matter: 中文摘要、英文摘要、目录、符号说明
%%   main matter:  正文、参考文献
%%   appendix:     附录
%%   backmatter:   致谢、发表论文
%%
%% ---------------------------------------------------------------
%%
%% 这里是按研究生论文的顺序安排的
%% 本科生论文请自行调整顺序
%%
%% ---------------------------------------------------------------

\maketitle

\frontmatter

\begin{abstract}
本文档是中国科学技术大学学位论文 \LaTeX{} 模板的一个示例文档. 这里会给出模板使用方法的简介. 模板选项及详细用法请参考说明文档 \verb|ustcthesis.pdf|.

\keywords{中国科学技术大学, 学位论文, \LaTeX{} 模板.}

\end{abstract}

\begin{enabstract}

This is a sample document of USTC thesis \LaTeX{} template. This document will show the usage of the template. For more information, please refer to the template document \verb|ustcthesis.pdf|.

\enkeywords{University of Science and Technology of China (USTC), Thesis, \LaTeX{} Template.}

\end{enabstract}

\tableofcontents
\listoffigures
\listoftables
\begin{notation}

\begin{tabbing}

  \hspace*{8em}\=\kill

  \textsf{集合} \\
  $\emptyset$  \> 空集 \\
  $\in$        \> 属于 \\
  $\subseteq$  \> 包含于 \\
  $\cap$       \> 交 \\
  $\cup$       \> 并 \\[\baselineskip]

  \textsf{数集} \\
  $\mathbb{N}$ \> 自然数集 \\
  $\mathbb{Z}$ \> 整数集 \\
  $\mathbb{Q}$ \> 有理数集 \\
  $\mathbb{R}$ \> 实数集 \\
  $\mathbb{C}$ \> 复数集 \\

\end{tabbing}

\end{notation}


\mainmatter

\chapter{绪论 (Introduction)}

\verb|ustcthesis| 是用于排版中国科学技术大学学位论文的 \LaTeX{} 模板. 适用于学士, 硕士和博士的学位论文编写. 由 jiaopjie 参照《中国科学技术大学研究生学位论文撰写规范》
\footnote{\url{http://gradschool.ustc.edu.cn/ylb/material/xw/wdxz/1.doc}}
和《关于本科毕业论文(设计)格式和统一封面的通知》
\footnote{\url{http://www.teach.ustc.edu.cn/document/doc-administration/4032.html}}
的要求编写. 本模板的编写过程中参考了已有的论文模板的编写方式.

早期的模板有中国科学技术大学本科论文模板 (作者XPS, 最后维护ywg)
\footnote{\url{https://github.com/ywgATustcbbs/ustcthesis.bachelor}}
和中国科学技术大学研究生论文模板 (作者Liuqs,主要维护Liuqs, Guolicai)
\footnote{\url{https://github.com/ywgATustcbbs/ustcthesis.msphd}}.
后来 ywg@USTC 对这两个模板进行了整合梳理并对其维护
\footnote{\url{https://github.com/ywgATustcbbs/ustcthesis}}.
该模板在研究生院网站和 bbs 站点均有下载链接. 2015 年, seisman 和 zepinglee 基于 \verb|ctex| 2.0 重新编写了模板
\footnote{\url{https://github.com/ustctug/ustcthesis}}.

ywg@USTC 整合后的模板有不少冗余代码. 该模板自 2016 年 2 月以来尚未更新. 并且在 TeXLive 2016 下, 该模板中的 \verb|appendix| 环境会报错. 而 seisman 和 zepinglee 写的模板中, 制作扉页时用了 \verb|tikz| 宏包. 不少写法个人不喜欢. 鉴于此我决定写个模板自用, 顺便把自己的编写思路记录下来. 因此本模板仅保证基本的格式要求. 额外的需求请自行使用相应的宏包.

\chapter{说明}
\label{chap:readme}

注意到《中国科学技术大学研究生学位论文撰写手册》中的格式要求有矛盾的地方.
这里适当作了一些调整.

\section{注意事项}

使用本模板之前请注意以下事项.

\begin{itemize}
  \item
    扉页中各项目与页面上边缘的距离作了调整.
  \item
    两到四字的章标题 (包括目录、摘要、致谢等), 字中间没有留空白.
  \item
    没有处理参考文献列表的格式. 请自行使用合适的 \verb|.bst| 格式文件.
  \item
    默认未开启伪粗体.
  \item
    模板支持使用 XeLaTeX 或者 LuaLaTeX 编译
    (推荐使用 XeLaTeX, 使用 LuaLaTeX 时不支持开启伪粗体).
  \item
    使用前请升级宏包. 其中 \verb|ctex| 宏包应该升级到 2.0 版本以上.
\end{itemize}

\section{文档选项}

模板新设置了一些文档选项, 如\autoref{tab:doc-option} 所示.

\begin{table}[!htb]
  \caption{模板的文档选项}
  \label{tab:doc-option}
  \centering
  \begin{tabular}{ll}
    \toprule
    选项                & 说明\\
    \midrule
    \verb|doctor|       & 博士学位 (默认)\\
    \verb|master|       & 硕士学位\\
    \verb|bachelor|     & 学士学位\\
    \midrule
    \verb|chinese|      & 中文论文 (默认)\\
    \verb|english|      & 英文论文\\
    \midrule
    \verb|academic|     & 学术学位 (默认)\\
    \verb|professional| & 专业学位\\
    \bottomrule
  \end{tabular}
\end{table}

另外, \verb|book| 文档类提供的文档选项仍然可用.
模板更改了其中 \verb|openright| 和 \verb|openany| 选项的默认值,
参见\autoref{tab:book-option}.

\begin{table}[!htb]
  \caption{单双面选项}
  \label{tab:book-option}
  \centering
  \begin{tabular}{lp{16em}l}
    \toprule
    选项             & 说明\\
    \midrule
    \verb|oneside|   & 单面格式\\
    \verb|twoside|   & 双面格式 (默认)\\
    \midrule
    \verb|openright| & 双面格式下新一章总是从奇数页开始\\
    \verb|openany|   & 双面格式下新一章总是从新一页开始
                       (双面格式下默认)\\
    \bottomrule
  \end{tabular}
\end{table}

\section{字体设置}

模板的中文字体由 \verb|ctex| 宏包自动设置.
以下几点需要注意.

\begin{itemize}
  \item
    较新的 Windows 系统下默认的微软雅黑可能不太适合打印.
  \item
    其他系统可能会缺少一些字体, 如仿宋、隶书以及 Times New Roman 等.
  \item
    \verb|ctex| 宏包默认不开启伪粗体, 此时宋体的加粗一般用黑体代替.
\end{itemize}

用户可以自定义文档字体.
相关命令可参见\autoref{tab:font-set}.
自定义字体时可通过 \verb|BoldFont| 选项设定粗体的替代字体
(或通过 \verb|AutoFakeBold| 选项开启伪粗体),
通过 \verb|ItalicFont| 选项设定斜体的替代字体.

\begin{table}[!htb]
  \caption{自定义字体命令}
  \label{tab:font-set}
  \centering
  \begin{tabular}{ll}
    \toprule
    命令                   & 说明\\
    \midrule
    \verb|\setmainfont|    & 设置衬线字体\\
    \verb|\setsansfont|    & 设置无衬线字体\\
    \verb|\setmonofont|    & 设置等宽字体\\
    \midrule
    \verb|\setCJKmainfont| & 设置中文衬线字体\\
    \verb|\setCJKsansfont| & 设置中文无衬线字体\\
    \verb|\setCJKmonofont| & 设置中文等宽字体\\
    \bottomrule
  \end{tabular}
\end{table}

下面针对 Windows 系统提供几点解决方案.

\begin{enumerate}
  \item
    使用 Fandol 字体 (比较容易缺字), 或者自行下载有粗体形式的字体
    (如思源字体). 可如下设置 (斜体用楷体代替).
\begin{verbatim}
\setCJKmainfont[ItalicFont={KaiTi}]{Source Han Serif SC}
\setCJKsansfont[ItalicFont={KaiTi}]{Source Han Sans SC}
\end{verbatim}
  \item
    使用中易字体, 并开启伪粗体. 伪粗体一般被认为质量比较差, 慎用.
\begin{verbatim}
\setCJKmainfont[AutoFakeBold=3,ItalicFont={KaiTi}]{SimSun}
\setCJKsansfont[AutoFakeBold=3,ItalicFont={KaiTi}]{SimHei}
\end{verbatim}
  \item
    使用中易黑体替换微软雅黑, 且关闭伪粗体.
    宋体的加粗形式一般是使用黑体代替.
\begin{verbatim}
\setCJKsansfont[ItalicFont={KaiTi}]{SimHei}
\end{verbatim}
    这种情况下, 中英混排会有以下两个小问题.
    \begin{itemize}
      \item
        黑体加粗时 (如扉页中的文章标题和\Emph{研究生论文}的章标题),
        中文不加粗, 而英文加粗, 稍微欠协调.
      \item
        宋体加粗时 (如\Emph{研究生论文}的插图和表格标题),
        中文是黑体, 而英文是加粗的 Times New Roman, 稍微欠协调.
    \end{itemize}
    用户可自行把相应的标题都修改为黑体不加粗.
    这主要包括中英文标题、章标题、插图和表格标题、“关键词”字样.
    其中, 中英文标题不加粗, 可在 \verb|\title| 和 \verb|\entitle| 的参数中
    加上 \verb|\mdseries|.
\begin{verbatim}
\ctexset{chapter={format+=\mdseries}}
\captionsetup{font+={md,sf}}
\entitle{\mdseries <English title>}
\end{verbatim}
\end{enumerate}

\section{扉页}

扉页中各项目与页面上边缘的距离作了调整.
下面是一些注意事项.

\begin{itemize}
  \item
    扉页所需的个人信息由\autoref{tab:info} 中命令设置, 它们应该用在导言区.
  \item
    若导师多于一人, 请一并用 \verb|\supervisor| 和 \verb|\ensupervisor| 给出.
  \item
    若英文专业或导师的文本过长, 可用 \verb|\\| 在合适的地方换行.
  \item
    扉页由命令 \verb|\maketitle| 生成, 它应该作为正文开始后的第一个命令.
  \item
    默认日期为当前日期.
\end{itemize}

\begin{table}[!htb]
  \caption{个人信息命令}
  \label{tab:info}
  \centering
  \begin{tabular}{lll}
    \toprule
    命令            & 命令 (英文)       & 说明\\
    \midrule
    \verb|\title|        & \verb|\entitle|        & 论文标题\\
    \verb|\author|       & \verb|\enauthor|       & 作者姓名\\
    \verb|\major|        & \verb|\enmajor|        & 学科专业\\
    \verb|\supervisor|   & \verb|\ensupervisor|   & 导师姓名\\
    \verb|\date|         & \verb|\endate|         & 完成日期\\
    \verb|\secrettext|   & \verb|\ensecrettext|   & 密级信息\\
    \bottomrule
  \end{tabular}
\end{table}

\verb|ctex| 宏包汉化了 \verb|\today| 命令,
并提供了 \verb|small|, \verb|big|, \verb|old| 三种样式.
\begin{itemize}
  \item \verb|small| 中文样式, 其中数字使用阿拉伯数字.
  \item \verb|big| 全汉字样式.
  \item \verb|old| 原英文样式
\end{itemize}
\verb|ctex| 默认的是 \verb|small| 样式. \verb|\ctexset{today=old}| 则切换到英文日期样式.

\section{模板提供的环境}

本模板对中英文摘要、符号说明、致谢、研究成果部分提供了专门的环境,
如\autoref{tab:doc-environment} 所示.
这主要是因为这些部分的格式要求与普通章的格式略有区别.

如果不在意这种区别的话, 完全可以用 \verb|\chapter*| 甚至 \verb|\chapter| 命令来代替.
例如 \verb|\chapter{摘要}|, \verb|\chapter{符号说明}|.

\begin{table}[!htb]
  \caption{模板提供的环境}
  \label{tab:doc-environment}
  \centering
  \begin{tabular}{ll}
    \toprule
    环境或命令              & 说明\\
    \midrule
    \verb|abstract|         & 中文摘要\\
    \verb|enabstract|       & 英文摘要\\
    \verb|notation|         & 符号说明\\
    \verb|acknowledgements| & 致谢\\
    \verb|publications|     & 研究成果\\
    \midrule
    \verb|\keywords|        & 中文关键词 (用在 \verb|abstract| 环境中)\\
    \verb|\enkeywords|      & 英文关键词 (用在 \verb|enabstract| 环境中)\\
    \bottomrule
  \end{tabular}
\end{table}

\section{图片和表格}

图片和表格很类似, 都可以混排的正文中, 也都可以放入对应的浮动体环境.
需要注意的是, 浮动体中 \verb|\caption| 命令应该置于 \verb|\label| 命令之前.

对于尺寸较小的图片和表格, 排版安排可以稍微随意一些.
即使混排在正文中也不会带来糟糕的影响.
比如\autoref{sec:bib}中没有表头的小表格就是用 \verb|center| 环境混排在正文中.

对于占用空间较大的图片和表格, 则应该放入对应的浮动体环境 \verb|figure| 和 \verb|table| 中.
这两个环境均可以带位置选项 \verb|h, t, b, p| 分别对应当前位置、页面顶端、页面底部、独占一页.
可以使用 \verb|!| 表示相对更靠近当前位置.

\subsection{图片}

\LaTeX{} 可以插入各种常见格式的图片.

\autoref{fig:test} 放在了浮动环境 \verb|figure| 中.
由于占用空间很小, 它很有可能就在下面.

\begin{figure}[!htb]
\centering
\includegraphics[height=3cm]{figure/test}
\caption{测试图片}
\label{fig:test}
\captionnote{这是个小图片. 该图片来自网络, 请勿用于商业用途.}
\end{figure}

\autoref{fig:test-big} 放在了浮动环境 \verb|figure| 中.
但由于占用空间较大, 它很有可能会跑到其他地方去了.

\begin{figure}[!htbp]
\centering
\includegraphics[width=8cm]{figure/test}
\caption{尺寸较大的图片}
\label{fig:test-big}
\captionnote{这是个尺寸较大的图片. 该图片放在了 \texttt{figure} 浮动环境中. 它的位置可以浮动. 它很有可能不会出现在代码所在的位置. 由于使用了 \texttt{p} 选项, 它很有可能独占了一页. 该图片来自网络, 请勿用于商业用途.}
\end{figure}

\subsection{表格}

\autoref{tab:test} 放在了浮动环境 \verb|table| 中.
由于占用空间很小, 它很有可能就在下面.

\begin{table}[!htb]
  \caption{普通表格}
  \label{tab:test}
  \centering
  \begin{tabular}{ccc}
    \toprule
    左 & 中 & 右\\
    \midrule
    1  & a  & b\\
    2  & a  & b\\
    3  & a  & b\\
    \bottomrule
  \end{tabular}
  \captionnote{这是个普通表格.}
\end{table}

\section{参考文献}
\label{sec:bib}

参考文献引用部分使用了 \verb|natbib| 宏包.
该宏包修改了 \verb|\cite| 命令, 并提供了新命令 \verb|\citet| 和 \verb|\citep|.

\verb|natbib| 宏包提供了两种引用模式: author-year 和 numerical.
它们可分别使用 \verb|authoryear| 和 \verb|numbers| 选项激活.
numerical 模式还有上标的形式, 可由 \verb|super| 选项激活.
其中 \verb|authoryear| 为默认选项.

宏包预定义了圆括号和方括号两种定界符, 分别使用 \verb|round| 和 \verb|square| 选项激活.
其中 \verb|round| 为默认选项.

本模板载入该宏包时使用了 \verb|numbers| 和 \verb|square| 选项.

命令 \verb|\bibliographystyle| 用于指定参考文献的样式. 它的参数是要使用参考文献样式对应的 \verb|.bst| 文件的文件名 (不包括扩展名). 宏包 \verb|natbib| 的作者提供了三个可用于 \verb|authoryear| 模式的 \verb|.bst| 文件: \verb|plainnat.bst|, \verb|abbrvnat.bst|, \verb|unsrtnat.bst|.

需要注意的是, 一些标准的 \verb|.bst| 文件 (如 \verb|plain.bst|) 只支持 numerical 模式.
在 author-year 模式下使用这些文件格式可能会遇到下述的错误信息.\\
\verb|Bibliography not compatible with author-year citations.|\\
此时只使用 numerical 模式即可.

如果使用的 \verb|.bst| 文件支持 author-year 模式,
在行文中是可以随时通过 \verb|\setcitestyle| 命令切换引用模式的.
\begin{itemize}
  \item
    \verb|\setcitestyle{authoryear,round}| 切换到 author-year 模式,
    并选定圆括号作为定界符.
  \item
    \verb|\setcitestyle{numbers,square}| 切换到 numerical 模式,
    并选定方括号作为定界符.
  \item
    \verb|\setcitestyle{super}| 切换到上标的 numerical 模式,
    不更改定界符.
\end{itemize}

研究生论文撰写手册要求, 对多作者的文献进行缩写时区分中英文文献的缩写词后缀. 这可能需要自己编写新的 \verb|.bst| 文件. 用户如有这样的要求的话, 可自行编写或下载满足条件的 \verb|.bst| 文件
\footnote{例如: \url{https://github.com/ustctug/gbt-7714-2015}}.

\subsection{numerical 模式}

命令 \verb|\setcitestyle{numbers,square}| 切换到 numerical 模式, 并选定方括号作为定界符.
此时三个引用命令的效果如下表所示.

\setcitestyle{numbers,square}
\begin{center}
  \begin{tabular}{ll}
    \toprule
    命令                      & 效果\\
    \midrule
    \verb|\cite{Knuth1986a}|  & \cite{Knuth1986a}\\
    \verb|\citet{Knuth1986a}| & \citet{Knuth1986a}\\
    \verb|\citep{Knuth1986a}| & \citep{Knuth1986a}\\
    \bottomrule
  \end{tabular}
\end{center}

在 \verb|natbib| 宏包中, 这些引用命令都有两个可选参数, 分别是引用的前缀和后缀.
例如, 命令 \verb|\cite| 在 \verb|numbers| 模式下的效果如下.\\
\verb|\cite[参见][第~1~章]{Knuth1986a}|
$\Rightarrow$
\cite[参见][第~1~章]{Knuth1986a}

\subsection{上标的 numerical 模式}

命令 \verb|\setcitestyle{super}| 切换到上标的 numerical 模式, 不更改定界符.
此时三个引用命令的效果如下表所示.

\setcitestyle{super}
\begin{center}
  \begin{tabular}{ll}
    \toprule
    命令                      & 效果\\
    \midrule
    \verb|\cite{Knuth1986a}|  & \cite{Knuth1986a}\\
    \verb|\citet{Knuth1986a}| & \citet{Knuth1986a}\\
    \verb|\citep{Knuth1986a}| & \citep{Knuth1986a}\\
    \bottomrule
  \end{tabular}
\end{center}

\subsection{author-year 模式}

命令 \verb|\setcitestyle{authoryear,round}| 切换到 author-year 模式, 并选定圆括号作为定界符.
此时三个引用命令的效果如下表所示.

\setcitestyle{authoryear,round}
\begin{center}
  \begin{tabular}{ll}
    \toprule
    命令                      & 效果\\
    \midrule
    \verb|\cite{Knuth1986a}|  & \cite{Knuth1986a}\\
    \verb|\citet{Knuth1986a}| & \citet{Knuth1986a}\\
    \verb|\citep{Knuth1986a}| & \citep{Knuth1986a}\\
    \bottomrule
  \end{tabular}
\end{center}

引用命令在 author-year 模式下会对多作者的文献使用 ``author1 et al.'' 的形式进行缩写.
而带星号的引用命令则会罗列所有作者. 这些命令也都可以一次引用多个文献.
使用效果参见\autoref{tab:cite-multi-author}.

\begin{table}[!htb]
  \caption{文献引用效果}
  \label{tab:cite-multi-author}
  \centering
  \begin{tabular}{lp{0.45\textwidth}l}
    \toprule
    命令                                   & 效果\\
    \midrule
    \verb|\cite{Mittelbach2004}|           & \cite{Mittelbach2004}\\
    \verb|\cite*{Mittelbach2004}|          & \cite*{Mittelbach2004}\\
    \verb|\cite{Knuth1984,Knuth1986a}|     & \cite{Knuth1984,Knuth1986a}\\
    \verb|\cite{Knuth1984,Mittelbach2004}| & \cite{Knuth1984,Mittelbach2004}\\
    \verb|\cite{Liu2013,Deng2001}|         & \cite{Liu2013,Deng2001}\\
    \bottomrule
  \end{tabular}
\end{table}

我们指出以下两点注意事项.
\begin{itemize}
  \item
    如果行文中使用了 author-year 模式, 则在 \verb|\bibliography| 命令之前应该切换回 numerical 模式. 否则参考文献列表前面不会有数字编号的前缀.
  \item
    如果使用了 \verb|.bib| 文件生成参考文献列表, 则列表中的每一条文献都需要在正文中使用 \verb|\cite| 等命令引用. 否则应该使用命令 \verb|\nocite| 声明.
\end{itemize}

\chapter{其他设置}
\label{chap:other}

\section{项目编号}

\verb|book| 文档类默认的 \verb|itemize| 与 \verb|enumerate| 环境的各条项目之间的距离偏大. 可以通过宏包 \verb|enumitem| 来进行个性化设置. 相关使用方法请参阅该宏包的说明文档.

下面的设置是一个简单的例子, 含义如下.
\begin{enumerate}
  \item
    项目之间的距离就等于普通的行间距.
  \item
    第一级环境的项目标号的缩进等于段落首行缩进.
  \item
    第一级 \verb|enumerate| 环境的项目标号样式为直立的 (1).
\end{enumerate}

\begin{verbatim}
\setlist{nolistsep, leftmargin=*}
\setlist[1]{labelindent=\parindent}
\setlist[enumerate,1]{label=\textup{(\arabic*)}}
\end{verbatim}

\section{自动引用}

自动引用可以由 \verb|hyperref| 宏包提供的命令 \verb|\autoref| 来实现.
该命令会在引用标号前自动加上对应的前缀, 并且把前缀和标号一起加上超链接.

模板对自动引用前缀作了一些处理.
使用效果可参见\autoref{chap:readme}.

需要注意的是, 该机制通过识别计数器工作.
数学中各种定理环境经常共用一个计数器. 这样, 相应的自动引用前缀就是一样的.
若要对不同定理环境区分前缀, 可通过 \verb|aliascnt| 宏包进行修正.
详情可参见这两个宏包的说明文档, 也可参见\autoref{sec:theorem}.

\section{定理类环境}
\label{sec:theorem}

为了适配中文习惯. 可以使用 \verb|amsthm| 宏包提供的 \verb|\newtheoremstyle| 命令来给出新的定理样式.
这里给出 ``定理'' 和 ``定义'' 两种定理样式.

\begin{verbatim}
\newtheoremstyle{ustctheorem}
  {\topsep}{\topsep}{\itshape}{\parindent}{\sffamily}{}{1em}{}
\newtheoremstyle{ustcdefinition}
  {\topsep}{\topsep}{}{\parindent}{\sffamily}{}{1em}{}
\end{verbatim}

为了配合 \verb|hyperref| 宏包自动引用功能, 需要通过宏包 \verb|aliascnt| 修正.

下面的代码定义了 ``定理'' 与 ``定义'' 两个环境, 它们共享一个计数器.
可以看到, \autoref{def:sphere} 与\autoref{thm:CRT} 的编号是连着的, 它们的自动引用也没有问题.

\begin{verbatim}
\theoremstyle{ustctheorem}
\newtheorem{theorem}{定理}[section]
\theoremstyle{ustcdefinition}
\newaliascnt{definition}{theorem}
\newtheorem{definition}[definition]{定义}
\aliascntresetthe{definition}
\newcommand\definitionautorefname{定义}
\end{verbatim}

在中文状态下, 可对 \verb|proof| 环境修改 \verb|\proofname| 来打印“证”字.
\begin{verbatim}
\renewcommand\proofname{\upshape\sffamily\indent 证.}
\end{verbatim}

\begin{definition}\label{def:sphere}
  $n$ 维\Emph{单位球面} (unit sphere) 是指 $n$ 维欧式空间 $\boldsymbol{E}^n$ 的子集
  \[
    S^n :=
    \left\{
    (x_1, x_2, \dots, x_n)
    \middle|
    \sum_{i=1}^n x_i^2 = 1
    \right\}.
  \]
\end{definition}

\begin{theorem}[中国剩余定理]\label{thm:CRT}
  设 $m = m_1 \cdots m_n$, 其中 $m_1, \dots, m_n$ 两两互素. 则同余方程组
  \begin{equation}
    \left\{
    \begin{matrix}
      x \equiv a_1 \mod m_1, \\
      \cdots \\
      x \equiv a_n \mod m_n, \\
    \end{matrix}
    \right.
  \end{equation}
  必有解, 且全部解为模 $m$ 的一个同余类 (congruence class).
\end{theorem}

\begin{proof}
  略.
\end{proof}

\begin{remark}\label{rem:CRT}
  中国剩余定理又称\Emph{孙子定理}, 最初起源于《孙子算经》中的问题:
  \begin{quote}
    \itshape
    今有物不知其数, 三三数之余二, 五五数之余三, 七七数之余二, 问物之几何?
  \end{quote}
\end{remark}

\section{长表格}

如果表格的长度超过了一页, 可使用 \verb|longtable| 宏包提供的 \verb|longtable| 环境.
还可以使用 \verb|booktabs| 宏包提供的更丰富的表格线.
该宏包提供的表格线也可以应用到 \verb|tabular| 生成的表格中.
参见\autoref{chap:readme}中的表格.

下表是一个长表格, 它不能放置在浮动环境中, 因此位置是准确的.
这里给它加了表头, 方便在其他位置引用.

\begin{center}
  \begin{longtable}{cccc}
    \caption{长表格}\label{tab:test-long}\\ % 首页的表序
    \toprule
    左 & 中 & 中 & 右\\
    \midrule
    \endfirsthead  % 到这里为止是首页的表头
    \caption[]{长表格 (续)}\\ % 后续页的表序
    \toprule
    左 & 中 & 中 & 右\\
    \midrule
    \endhead  % 到这里为止是后续页的表头
    \hline
    \multicolumn{4}{r}{\small 续下页}
    \endfoot  % 到这里为止是首页的表尾
    \bottomrule
    \captionnote{这是个长表格.}
    \endlastfoot  % 到这里为止是后续页的表尾
    1  &  abc  &  def  &  xyz \\
    2  &  abc  &  def  &  xyz \\
    3  &  abc  &  def  &  xyz \\
    4  &  abc  &  def  &  xyz \\
    5  &  abc  &  def  &  xyz \\
    6  &  abc  &  def  &  xyz \\
    7  &  abc  &  def  &  xyz \\
    8  &  abc  &  def  &  xyz \\
    9  &  abc  &  def  &  xyz \\
    10 &  abc  &  def  &  xyz \\
    11 &  abc  &  def  &  xyz \\
    12 &  abc  &  def  &  xyz \\
    13 &  abc  &  def  &  xyz \\
    14 &  abc  &  def  &  xyz \\
    15 &  abc  &  def  &  xyz \\
    16 &  abc  &  def  &  xyz \\
    17 &  abc  &  def  &  xyz \\
    18 &  abc  &  def  &  xyz \\
    19 &  abc  &  def  &  xyz \\
    20 &  abc  &  def  &  xyz \\
    21 &  abc  &  def  &  xyz \\
    22 &  abc  &  def  &  xyz \\
    23 &  abc  &  def  &  xyz \\
    24 &  abc  &  def  &  xyz \\
    25 &  abc  &  def  &  xyz \\
    26 &  abc  &  def  &  xyz \\
    27 &  abc  &  def  &  xyz \\
    28 &  abc  &  def  &  xyz \\
    29 &  abc  &  def  &  xyz \\
    30 &  abc  &  def  &  xyz \\
    31 &  abc  &  def  &  xyz \\
    32 &  abc  &  def  &  xyz \\
    33 &  abc  &  def  &  xyz \\
    34 &  abc  &  def  &  xyz \\
    35 &  abc  &  def  &  xyz \\
  \end{longtable}
\end{center}


\setcitestyle{numbers}
\bibliography{bib/latex}

\appendix

\chapter{关于本科毕业论文(设计)格式和统一封面的通知}

\noindent
各院系:

为加强本科毕业论文的管理,提高论文质量,同时规范全校本科毕业论文(设计)格式,现对本科毕业论文格式和统一封面规定如下:
\setlist[enumerate,1]{label={\arabic*.}}
\setlist[enumerate,2]{label={(\arabic*)}}
\setlist[enumerate,3]{label={\alph*.}}
\begin{enumerate}
  \item
    本科毕业论文按编排顺序应包括以下内容:封面、扉页、致谢、目录、中文内容摘要、英文内容摘要、正文章节、参考文献或资料注释、附录等。
  \item
    本科毕业论文的格式要求:
    \begin{enumerate}
      \item
        封面中“论文题目”等内容用四号宋体。
      \item
        除封面、扉页外,每面上部加页眉,用小 5 号字标注“中国科学技术大学本科毕业论文”,居中。
      \item
        从目录页开始在每面底部居中用小五宋体连续编页码。
      \item
        论文的“致谢”、“目录”等标题用小二号黑体字,居中。目录一般列三级,后附规范的页号。正文中的标题分章、节、段三级;章、节标题居中,段标题居左,分别用三号黑体、小三黑体、四号黑体。具体内容用小四号宋体,每行间距为22磅,科学公式和符号要符合国标,公式要单独占行、居中、行距为单倍行距。表格、插图全文要分别统一编号或按章编号,标题用小四宋体:(表格标题居表上方,插图标题居图下方),居中。
      \item
        参考文献的内容包括:序号、作者名、书名或文章名、刊物名或出版社名、刊物期卷、页和日期,用小四宋体,外文期刊名用白斜体。
      \item
        附录为:
        \begin{enumerate}
          \item 重要参考文献中相关内容和章节复印件;
          \item 作者或导师所做的与本论文有关的成果复印件。
          \item[] 要求用A4纸复印附于参考文献后。
        \end{enumerate}
    \end{enumerate}
  \item
    本科毕业论文(设计)封面学校统一印制。
  \item
    装订要求:每份论文必须用A4纸打印(复印)、装订成册。
  \item
    具体格式详见附件式样。
  \item[]
    附件:%
    \href{https://www.teach.ustc.edu.cn/wp-content/uploads/legacy/documents/2015/bylwgs2015f.doc} {本科毕业论文(设计)式样}
\end{enumerate}

\begin{flushright}
中国科学技术大学教务处\\
二〇一五年五月六日
\end{flushright}


\backmatter

\begin{acknowledgements}

早期的模板有中国科学技术大学本科论文模板 (作者XPS, 最后维护ywg)
和中国科学技术大学研究生论文模板 (作者Liuqs,主要维护Liuqs, Guolicai).
后来 ywg@USTC 对这两个模板进行了整合梳理并对其维护.
该模板在研究生院网站和 bbs 站点均有下载链接. 2015 年, seisman 和 zepinglee 基于 \verb|ctex| 2.0 重新编写了模板.

本模板参考了 ywg@USTC 以及 seisman 与 zepinglee 编写的模板. 感谢相关模板的作者以及维护者. 也感谢更早版本的模板作者以及维护者.

\vskip\baselineskip

\begin{flushright}
jiaopjie

2017 年春于合肥
\end{flushright}

\end{acknowledgements}

\begin{publications}

\section*{已发表论文}

\begin{enumerate}
\renewcommand\labelenumi{[\arabic{enumi}]}
\item A A A A A A A A A
\end{enumerate}

\section*{待发表论文}

\begin{enumerate}
\renewcommand\labelenumi{[\arabic{enumi}]}
\setcounter{enumi}{1}
\item B B B B B B B B B
\end{enumerate}

\section*{研究报告}

\begin{enumerate}
\renewcommand\labelenumi{[\arabic{enumi}]}
\setcounter{enumi}{2}
\item C C C C C C C C C
\end{enumerate}

\end{publications}


\end{document}
%% ---------------------------------------------------------------
