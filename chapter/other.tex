\chapter{其他设置}
\label{chap:other}

\section{项目编号}

\verb|book| 文档类默认的 \verb|itemize| 与 \verb|enumerate| 环境的各条项目之间的距离偏大. 可以通过宏包 \verb|enumitem| 来进行个性化设置. 相关使用方法请参阅该宏包的说明文档.

下面的设置是一个简单的例子, 含义如下.
\begin{enumerate}
  \item
    项目之间的距离就等于普通的行间距.
  \item
    第一级环境的项目标号的缩进等于段落首行缩进.
  \item
    第一级 \verb|enumerate| 环境的项目标号样式为直立的 (1).
\end{enumerate}

\begin{verbatim}
\setlist{nolistsep, leftmargin=*}
\setlist[1]{labelindent=\parindent}
\setlist[enumerate,1]{label=\textup{(\arabic*)}}
\end{verbatim}

\section{自动引用}

自动引用可以由 \verb|hyperref| 宏包提供的命令 \verb|\autoref| 来实现.
该命令会在引用标号前自动加上对应的前缀, 并且把前缀和标号一起加上超链接.

模板对自动引用前缀作了一些处理.
使用效果可参见\autoref{chap:readme}.

需要注意的是, 该机制通过识别计数器工作.
数学中各种定理环境经常共用一个计数器. 这样, 相应的自动引用前缀就是一样的.
若要对不同定理环境区分前缀, 可通过 \verb|aliascnt| 宏包进行修正.
详情可参见这两个宏包的说明文档, 也可参见\autoref{sec:theorem}.

\section{定理类环境}
\label{sec:theorem}

为了适配中文习惯. 可以使用 \verb|amsthm| 宏包提供的 \verb|\newtheoremstyle| 命令来给出新的定理样式.
这里给出 ``定理'' 和 ``定义'' 两种定理样式.

\begin{verbatim}
\newtheoremstyle{ustctheorem}
  {\topsep}{\topsep}{\itshape}{\parindent}{\sffamily}{}{1em}{}
\newtheoremstyle{ustcdefinition}
  {\topsep}{\topsep}{}{\parindent}{\sffamily}{}{1em}{}
\end{verbatim}

为了配合 \verb|hyperref| 宏包自动引用功能, 需要通过宏包 \verb|aliascnt| 修正.

下面的代码定义了 ``定理'' 与 ``定义'' 两个环境, 它们共享一个计数器.
可以看到, \autoref{def:sphere} 与\autoref{thm:CRT} 的编号是连着的, 它们的自动引用也没有问题.

\begin{verbatim}
\theoremstyle{ustctheorem}
\newtheorem{theorem}{定理}[section]
\theoremstyle{ustcdefinition}
\newaliascnt{definition}{theorem}
\newtheorem{definition}[definition]{定义}
\aliascntresetthe{definition}
\newcommand\definitionautorefname{定义}
\end{verbatim}

在中文状态下, 可对 \verb|proof| 环境修改 \verb|\proofname| 来打印“证”字.
\begin{verbatim}
\renewcommand\proofname{\upshape\sffamily\indent 证.}
\end{verbatim}

\begin{definition}\label{def:sphere}
  $n$ 维\Emph{单位球面} (unit sphere) 是指 $n$ 维欧式空间 $\boldsymbol{E}^n$ 的子集
  \[
    S^n :=
    \left\{
    (x_1, x_2, \dots, x_n)
    \middle|
    \sum_{i=1}^n x_i^2 = 1
    \right\}.
  \]
\end{definition}

\begin{theorem}[中国剩余定理]\label{thm:CRT}
  设 $m = m_1 \cdots m_n$, 其中 $m_1, \dots, m_n$ 两两互素. 则同余方程组
  \begin{equation}
    \left\{
    \begin{matrix}
      x \equiv a_1 \mod m_1, \\
      \cdots \\
      x \equiv a_n \mod m_n, \\
    \end{matrix}
    \right.
  \end{equation}
  必有解, 且全部解为模 $m$ 的一个同余类 (congruence class).
\end{theorem}

\begin{proof}
  略.
\end{proof}

\begin{remark}\label{rem:CRT}
  中国剩余定理又称\Emph{孙子定理}, 最初起源于《孙子算经》中的问题:
  \begin{quote}
    \itshape
    今有物不知其数, 三三数之余二, 五五数之余三, 七七数之余二, 问物之几何?
  \end{quote}
\end{remark}

\section{长表格}

如果表格的长度超过了一页, 可使用 \verb|longtable| 宏包提供的 \verb|longtable| 环境.
还可以使用 \verb|booktabs| 宏包提供的更丰富的表格线.
该宏包提供的表格线也可以应用到 \verb|tabular| 生成的表格中.
参见\autoref{chap:readme}中的表格.

下表是一个长表格, 它不能放置在浮动环境中, 因此位置是准确的.
这里给它加了表头, 方便在其他位置引用.

\begin{center}
  \begin{longtable}{cccc}
    \caption{长表格}\label{tab:test-long}\\ % 首页的表序
    \toprule
    左 & 中 & 中 & 右\\
    \midrule
    \endfirsthead  % 到这里为止是首页的表头
    \caption[]{长表格 (续)}\\ % 后续页的表序
    \toprule
    左 & 中 & 中 & 右\\
    \midrule
    \endhead  % 到这里为止是后续页的表头
    \hline
    \multicolumn{4}{r}{\small 续下页}
    \endfoot  % 到这里为止是首页的表尾
    \bottomrule
    \captionnote{这是个长表格.}
    \endlastfoot  % 到这里为止是后续页的表尾
    1  &  abc  &  def  &  xyz \\
    2  &  abc  &  def  &  xyz \\
    3  &  abc  &  def  &  xyz \\
    4  &  abc  &  def  &  xyz \\
    5  &  abc  &  def  &  xyz \\
    6  &  abc  &  def  &  xyz \\
    7  &  abc  &  def  &  xyz \\
    8  &  abc  &  def  &  xyz \\
    9  &  abc  &  def  &  xyz \\
    10 &  abc  &  def  &  xyz \\
    11 &  abc  &  def  &  xyz \\
    12 &  abc  &  def  &  xyz \\
    13 &  abc  &  def  &  xyz \\
    14 &  abc  &  def  &  xyz \\
    15 &  abc  &  def  &  xyz \\
    16 &  abc  &  def  &  xyz \\
    17 &  abc  &  def  &  xyz \\
    18 &  abc  &  def  &  xyz \\
    19 &  abc  &  def  &  xyz \\
    20 &  abc  &  def  &  xyz \\
    21 &  abc  &  def  &  xyz \\
    22 &  abc  &  def  &  xyz \\
    23 &  abc  &  def  &  xyz \\
    24 &  abc  &  def  &  xyz \\
    25 &  abc  &  def  &  xyz \\
    26 &  abc  &  def  &  xyz \\
    27 &  abc  &  def  &  xyz \\
    28 &  abc  &  def  &  xyz \\
    29 &  abc  &  def  &  xyz \\
    30 &  abc  &  def  &  xyz \\
    31 &  abc  &  def  &  xyz \\
    32 &  abc  &  def  &  xyz \\
    33 &  abc  &  def  &  xyz \\
    34 &  abc  &  def  &  xyz \\
    35 &  abc  &  def  &  xyz \\
  \end{longtable}
\end{center}
