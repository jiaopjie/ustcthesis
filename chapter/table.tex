\chapter{图片和表格}

\section{图片}

图~\ref{fig:test} 是个小图片. 因此尽管它在浮动环境中, 却很有可能就在下面.

\begin{figure}[!htb]
\centering
\includegraphics[height=4cm]{figure/test}
\caption{测试图片}
\label{fig:test}
\captionnote{这是个小图片. 该图片来自网络, 请勿用于商业用途.}
\end{figure}

图~\ref{fig:bigfigure} 是一张尺寸比较大的图片. 再加上这个图片是在一个浮动环境中, 因此它很有可能会跑到其他地方去.

\begin{figure}[!htb]
\centering
\includegraphics[width=8cm]{figure/test}
\caption{尺寸较大的图片}
\label{fig:bigfigure}
\captionnote{这是个比较大的图片. 该图片来自网络, 请勿用于商业用途. 该图片来自网络, 请勿用于商业用途. 该图片来自网络, 请勿用于商业用途. 该图片来自网络, 请勿用于商业用途.}
\end{figure}

\section{普通表格}

表~\ref{tab:test} 是个小表格. 因此尽管它在浮动环境中, 却很有可能就在下面.

\begin{table}[!htb]
  \caption{普通表格}
  \label{tab:test}
  \centering
  \begin{tabular}{ccc}
    \toprule
    左 & 中 & 右\\
    \midrule
    1  & a  & b\\
    2  & a  & b\\
    3  & a  & b\\
    4  & a  & b\\
    5  & a  & b\\
    \bottomrule
  \end{tabular}
  \captionnote{这是个普通表格.}
\end{table}

\section{长表格}

表~\ref{tab:longtable} 是一个长表格. 长表格似乎不能放在浮动环境中. 因此, 它的位置会是准确的.

\begin{center}
  \begin{longtable}{cccc}
    \caption{长表格}\label{tab:longtable}\\ % 首页的表序
    \toprule
    左 & 中 & 中 & 右\\
    \midrule
    \endfirsthead  % 到这里为止是首页的表头
    \caption[]{长表格 (续)}\\ % 后续页的表序
    \toprule
    左 & 中 & 中 & 右\\
    \midrule
    \endhead  % 到这里为止是后续页的表头
    \hline
    \multicolumn{4}{r}{\small 续下页}
    \endfoot  % 到这里为止是首页的表尾
    \bottomrule
    \captionnote{这是个长表格.}
    \endlastfoot  % 到这里为止是后续页的表尾
    1  &  abc  &  def  &  xyz \\
    2  &  abc  &  def  &  xyz \\
    3  &  abc  &  def  &  xyz \\
    4  &  abc  &  def  &  xyz \\
    5  &  abc  &  def  &  xyz \\
    6  &  abc  &  def  &  xyz \\
    7  &  abc  &  def  &  xyz \\
    8  &  abc  &  def  &  xyz \\
    9  &  abc  &  def  &  xyz \\
    10 &  abc  &  def  &  xyz \\
    11 &  abc  &  def  &  xyz \\
    12 &  abc  &  def  &  xyz \\
    13 &  abc  &  def  &  xyz \\
    14 &  abc  &  def  &  xyz \\
    15 &  abc  &  def  &  xyz \\
    16 &  abc  &  def  &  xyz \\
    17 &  abc  &  def  &  xyz \\
    18 &  abc  &  def  &  xyz \\
    19 &  abc  &  def  &  xyz \\
    20 &  abc  &  def  &  xyz \\
    21 &  abc  &  def  &  xyz \\
    22 &  abc  &  def  &  xyz \\
    23 &  abc  &  def  &  xyz \\
    24 &  abc  &  def  &  xyz \\
    25 &  abc  &  def  &  xyz \\
    26 &  abc  &  def  &  xyz \\
    27 &  abc  &  def  &  xyz \\
    28 &  abc  &  def  &  xyz \\
    29 &  abc  &  def  &  xyz \\
    30 &  abc  &  def  &  xyz \\
    31 &  abc  &  def  &  xyz \\
    32 &  abc  &  def  &  xyz \\
    33 &  abc  &  def  &  xyz \\
    34 &  abc  &  def  &  xyz \\
    35 &  abc  &  def  &  xyz \\
  \end{longtable}
\end{center}

\subsection{项目编号}

\verb|book| 文档类默认的 \verb|itemize| 与 \verb|enumerate| 环境的各条项目之间的距离偏大. 可以通过宏包 \verb|enumitem| 来进行个性化设置. 相关使用方法请参阅该宏包的说明文档.

下面给出的是一个例子. 其中第一行设置项目之间的距离就等于普通的行间距. 第二行表示对 \verb|enumerate| 环境的第一级进行设置. 第三行表示项目标号的缩进等于段落首行缩进, 并且标号使用直立字体, 标号格式是阿拉伯数字外加圆括号.
\begin{verbatim}
\setlist{nolistsep}
\setlist[enumerate,1]
  {labelindent=\parindent, leftmargin=*, label=\textup{(\arabic*)}}
\end{verbatim}
